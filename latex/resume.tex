\documentclass[12pt, twoside]{report}  
\usepackage[utf8]{inputenc}
\usepackage[a4paper, width=170mm,top=25mm,bottom=15mm]{geometry}


\usepackage{graphicx}    
\graphicspath{{figs/}} 

\usepackage{float}
\usepackage{caption}
\usepackage{subcaption}  
\usepackage{amsmath}  
\usepackage{amsfonts}
\usepackage{amssymb}   
\usepackage{amsthm}
\usepackage{thmtools}
\usepackage{indentfirst} 
\usepackage{url}
\usepackage{hyperref}
\usepackage{wrapfig}
\usepackage[dvipsnames]{xcolor}
\usepackage{titlesec, color}
\usepackage{ifthen}
\usepackage{etoolbox}


\input{config/settings.tex}


%title
\title{Élement de recherche opérationnelle}
\subtitle{Déneiger Montréal}
\author{Marc Plantevit}
\date{\today}

%keyword

%logo
\logo{./figs/Epita.png}
%\secondlogo{figs/logo/su.png}  % optional field

%supervisors - each one of these fields are optional
\firstsupervisor{Tristan Denis}
\firstsupervisorrole{Drone}

\secondsupervisor{Mathis Guilbaud}
\secondsupervisorrole{Drone}

\thirdsupervisor{Lea Cruciani}
\thirdsupervisorrole{Chasse neige}

\fourthsupervisor{Alexandre Gautier}
\fourthsupervisorrole{Chasse neige}

\fifthsupervisor{Adrien Lorge}
\fifthsupervisorrole{Chasse neige}


\begin{document}

\input{config/titlepage.tex}

\newgeometry{width=140mm,top=25mm,bottom=25mm}
\tableofcontents


\newgeometry{width=170mm,top=35mm,bottom=35mm}
\pagestyle{plain}

\chapter{Introduction}\label{chp:intro}

Dans le cadre des fortes chutes de neiges à Montréal, il nous a été demandé de
trouver une solution pour déneiger efficacement la ville. Pour cela, nous avons
à notre disposition un drone, qui nous sert à déterminer les zones à déneiger, 
ainsi que qu'un nombre limité de matériel et de personnel.\\
Afin de résoudre ce problème, nous travaillerons avec une représentation sous
forme de graphe de la ville.

\chapter{Le drone}\label{chp:drone}
\section{Problèmes}
Il nous a donc fallu trouver une méthode pour que le drone vérifie efficacement
l'état de toutes les routes de Montréal.
\section{Première solution}


Nous sommes partis du principe que le drone ne pouvait déterminer l'état d'une
route que s'il la parcourait entièrement. La méthode que nous avons privilégié
est celle du chemin Hamiltonien (chemin passant par toutes les arrêtes d'un
graphe). Dans le cas où notre graphe ne contenait pas de chamin Hamiltonien, 
nous rajoutions des arrêtes en partant du principe que le drone peut voler au
dessus des bâtiments.
\subsection{Les limites}


Premièrement, nous n'utilisons pas toutes les ressources du drone. En effet,
celui-ci peut analyser les rues sans avoir à les survoler directement. Cela
induit donc une distance parcourue inutilement élevée, et donc une perte de
temps et d'argent conséquente.\\
De plus, trouver un chemin Hamiltonien dans un graphe est très coûteux et
la solution de rajouter des arrêtes complexifie encore plus le problème.


\section{Deuxième solution}


Pour cette solution, nous sommes parties du principe que le drone peut analyser
les rues simplement en étant à une de ses extrémité. Cette solution s'apparente
à trouver une solution au problème de couverture de graphe par sommets. Pour
trouver une solution viable nous 
\bibliographystyle{ieeetr}  
\bibliography{sources}

\appendix
\chapter{Your appendix}\label{apx:morestuff}
Aenean sed dapibus quam. 
Sed aliquam mattis varius. 
Vestibulum tortor dui, interdum ut nisi sit amet, feugiat pharetra dolor. 
Suspendisse faucibus, purus sit amet mollis volutpat, tellus arcu condimentum arcu, sit amet ullamcorper dui orci ac erat. 
Vestibulum imperdiet ex quis nunc fringilla, rutrum viverra libero finibus. 
Pellentesque blandit elit est, a venenatis ligula vulputate ut. 
Phasellus in ipsum est. Donec aliquam ligula tristique enim condimentum, sed fringilla augue interdum. 
Pellentesque accumsan, orci at rutrum consequat, risus sapien scelerisque lectus, in dapibus leo nunc a est. Vestibulum libero tellus, tempor non facilisis eget, rutrum quis elit.
Proin suscipit nec massa at euismod. Vivamus aliquam diam bibendum laoreet auctor.
Proin sit amet ornare nunc, eget faucibus odio. 
Pellentesque pretium ac ante pharetra efficitur. 
Ut semper ipsum sed nisl finibus bibendum. Donec lobortis sagittis ex, in pretium est feugiat vitae. 


\end{document}
